\chapter{Общая информация}

Для класса определены четыре значения состояния компиляции:
\begin{itemize}
    \item \lstinline:final: --- финальная версия документа;
    \item \lstinline:monochrome: --- финальная версия документа с ЧБ текстом;
    \item \lstinline:build: --- промежуточная версия документа;
    \item \lstinline:draft: --- черновая версия документа.
\end{itemize}

При значениях \lstinline:build: и \lstinline:draft: на каждом листе выводится водяной знак
<<Черновик>>. Если параметр опущен (т.е. ничего не указано, как в случае
компиляции через MiKTeX), то он приравнивается к \lstinline:final:.

В классе определены следующие размеры шрифтов:
\begin{itemize}
    \item \lstinline:\LARGE: --- 20 pt;
    \item \lstinline:\Large: --- 18 pt;
    \item \lstinline:\large: --- 16 pt;
    \item \lstinline:\normalsize: --- 14 pt;
    \item \lstinline:\small: --- 12 pt;
    \item \lstinline:\footnotesize: --- 10 pt.
\end{itemize}

Для деления документа на части можно использовать следующие команды (перечислены
в порядке увеличения уровня вложенности):

\begin{itemize}
    \item \lstinline:\chapter{Название}: --- основные части, большие структурные единицы
    (уровень \textbf{Введение} и т.д.);
    \item \lstinline:\section{Название}: --- 2 уровень вложенности;
    \item \lstinline:\subsection{Название}: --- 3 уровень вложенности;
    \item \lstinline:\subsubsection{Название}: --- 4 уровень вложенности, не добавляется в
    содержание, не рекомендуется к использованию.
\end{itemize}

Большие структурные единицы, написанные в отдельном файле, рекомендуется
добавлять с помощью команды \lstinline:\include{./path/to/file}:, а более мелкие ---
с помощью команды \lstinline:\input{./path/to/file}:. 

Класс создан под компилятор \lstinline:xelatex: и \lstinline:biber: и работает
как на дистрибутивах TeX Live, так и на дистрибутиве MiKTeX. Список всех
подключенных пакетов можно найти в приложении~\ref{appendix:packages}.

Структура директории класса имеет следующий вид:
\vspace{0.5cm}
\begin{small}
\dirtree{%
.1 my\_latex\_class.
.2 bmstu-logo.pdf\DTcomment{Герб МГТУ}.
.2 my-abbreviations.sty\DTcomment{Сокращения}.
.2 my-appendix.sty\DTcomment{Приложения}.
.2 my-bibliography.sty\DTcomment{Литература}.
.2 my-bmstu-titlepage.sty\DTcomment{Титульный лист}.
.2 my-essay.sty\DTcomment{Реферат}.
.2 my-figures-tables.sty\DTcomment{Рисунки и таблицы}.
.2 my-listing.sty\DTcomment{Листинги}.
.2 my-toc.sty\DTcomment{Содержание}.
.2 my-latex-class.cls\DTcomment{Основные настройки}.
}
\end{small}
\vspace{0.5cm}

При необходимости настройки отдельных элементов класса можно изменить в
соответствующем файле.