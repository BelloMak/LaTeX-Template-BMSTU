\section{Рисунки}

В преамбуле документа стоит указать директорию, в которой предполагается
располагать изображения, с помощью команды 

\lstinline:\graphicspath{{./path/}{path}}:

Стоит обратить внимание, что может быть указано несколько расположений (в том
числе и сетевых). В дальнейшем пути до изображений необходимо писать от
указанной ранее точки.

! Расположение \lstinline:./my_latex_class/: необходимо для доступа к логотипу
МГТУ им. Н.Э.~Баумана, его наличие обязательно при создании титульной страницы. 

Далее приведены примеры использования команд для вставки картинок. В классе
определены две команды для вставки изображений:
\begin{enumerate}
    \item \lstinline:\img[position]{label}{path/to/file}{caption}{scale}: ---
    основная команда для вставки изображения;
    \item \lstinline:\rotateimg[position]{label}{path/to/file}{caption}{scale}: ---
    вставка повернутого на $90^\circ$ изображения.
\end{enumerate}
Где
\begin{itemize}
    \item \lstinline:position: --- необязательный параметр, отвечает за положение
    рисунка, может быть h, t, b и !h;
    \item \lstinline:label: --- метка для создания ссылок на изображение в тексте;
    \item \lstinline:path/to/file: --- путь до изображения;
    \item \lstinline:caption: --- подпись к рисунку;
    \item \lstinline:scale: --- число от $0$ до $1$, соответствующее желаемому
    масштабу изображения относительно заполняемой текстом ширины (в первом
    случае) или высоты (во втором случае) страницы.
\end{itemize}

! Вставка текста на страницу, содержащую повернутое изображение, не
поддерживается. 

На рисунке~\ref{fig:kittensmall} представлен красивый котенок, а
рисунок занимает половину ширины печатной области, а на рисунке~\ref{fig:kitten}
она занимает всю ширину печатной области и имеет подпись, вставленную с помощью
команды, определенной в файле \textit{001-captions.tex}. На
рисунке~\ref{fig:street} представлена панорама города и он повернут на
$90^\circ$ и занимает всю высоту печатной области. 

Ниже приведены команды, использованные для вставки картинок, и результаты их
работы. 

Команда\\
\lstinline$\img[!h]{fig:kittensmall}{kitten.jpg}$\\
\lstinline$            {Сгенерированный нейросетью котенок}{0.5}$

и ее результат (рисунок~\ref{fig:kittensmall}).

\img[!h]{fig:kittensmall}{kitten.jpg}{Сгенерированный нейросетью котенок}{0.5}

\newpage
Команда\\
\lstinline$\img[!h]{fig:kitten}{kittensmall}$\\
\lstinline$            {Сгенерированный нейросетью котенок}{1}$

и ее результат (рисунок~\ref{fig:kitten}).

\img[!h]{fig:kitten}{kitten.jpg}{Сгенерированный нейросетью котенок}{1}

Команда 

\lstinline$\rotateimg[!h]{fig:street}{street.jpg}{Панорама города}{1}$

и ее результат (рисунок~\ref{fig:street}).
\newpage
\rotateimg[!h]{fig:street}{street.jpg}{Панорама города}{1}
\clearpage